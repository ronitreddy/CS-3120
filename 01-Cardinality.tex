\documentclass[12pt]{article}
\usepackage[top=1in,bottom=1in,left=0.75in,right=0.75in,centering]{geometry}
\usepackage{fancyhdr}
\usepackage{epsfig}
\usepackage[pdfborder={0 0 0}]{hyperref}
\usepackage{palatino}
\usepackage{wrapfig}
\usepackage{lastpage}
\usepackage{color}
\usepackage{ifthen}
\usepackage[table]{xcolor}
\usepackage{graphicx,type1cm,eso-pic,color}
\usepackage{hyperref}
\usepackage{amsmath}
\usepackage{wasysym}
\usepackage{amsfonts}

\def\course{CS 3120: Discrete Math and Theory II}
\def\homework{Set Cardinality}
\def\semester{Fall 2024}

\newboolean{solution}
\setboolean{solution}{false}

% add watermark if it's a solution exam
% see http://jeanmartina.blogspot.com/2008/07/latex-goodie-how-to-watermark-things-in.html
\makeatletter
\AddToShipoutPicture{%
\setlength{\@tempdimb}{.5\paperwidth}%
\setlength{\@tempdimc}{.5\paperheight}%
\setlength{\unitlength}{1pt}%
\put(\strip@pt\@tempdimb,\strip@pt\@tempdimc){%
\ifthenelse{\boolean{solution}}{
\makebox(0,0){\rotatebox{45}{\textcolor[gray]{0.95}%
{\fontsize{5cm}{3cm}\selectfont{\textsf{Solution}}}}}%
}{}
}}
\makeatother

\pagestyle{fancy}

\fancyhf{}
\lhead{\course}
\chead{Page \thepage\ of \pageref{LastPage}}
\rhead{\semester}
%\cfoot{\Large (the bubble footer is automatically inserted into this space)}

\setlength{\headheight}{14.5pt}

\newenvironment{itemlist}{
\begin{itemize}
\setlength{\itemsep}{0pt}
\setlength{\parskip}{0pt}}
{\end{itemize}}

\newenvironment{numlist}{
\begin{enumerate}
\setlength{\itemsep}{0pt}
\setlength{\parskip}{0pt}}
{\end{enumerate}}

\newcounter{pagenum}
\setcounter{pagenum}{1}
\newcommand{\pageheader}[1]{
\clearpage\vspace*{-0.4in}\noindent{\large\bf{Page \arabic{pagenum}: {#1}}}
\addtocounter{pagenum}{1}
\cfoot{}
}

\newcounter{quesnum}
\setcounter{quesnum}{1}
\newcommand{\question}[2][??]{
\begin{list}{\labelitemi}{\leftmargin=2em}
\item [\arabic{quesnum}.] {} {#2}
\end{list}
\addtocounter{quesnum}{1}
}


\definecolor{red}{rgb}{1.0,0.0,0.0}
\newcommand{\answer}[2][??]{
\ifthenelse{\boolean{solution}}{
\color{red} #2 \color{black}}
{\vspace*{#1}}
}

\definecolor{blue}{rgb}{0.0,0.0,1.0}

\begin{document}

\section*{\homework}


\question[3]{
For each of the following claims, state whether it is true or false and then prove your assertion.
}

\begin{itemize}
	\item All finite sets have an \emph{injection} to $\mathbb{N}$ \textbf{True}. A finite set $S$ with $n$ elements can be injected into $\mathbb{N}$ by mapping each element $s_i \in S$ to its position $i$ in the sequence $1, 2, \ldots, n$. This function is clearly injective as no two different elements in $S$ map to the same natural number.
	\item All finite sets have a \emph{surjection} to $\mathbb{N}$
    \textbf{False.} A finite set has a finite number of elements, while $\mathbb{N}$ is infinite. A surjective function requires every element in $\mathbb{N}$ to be the image of some element in the finite set, which is impossible because there will always be natural numbers that are not mapped to.
	\item If $A$ is a countably infinite set (i.e., $|A|=|\mathbb{N}|$) and $B$ is a also a countably infinite set (i.e., $|B| = |\mathbb{N}|$), then $A \cup B$ is also countable.\textbf{True.} A and B are countably infinite sets, so the Union of these two sets A and B will produce another countably infinite set. For example, we can count them by mapping values in the natural numbers to alternating values in each set, counting $a1,b1,a2,b2, \ldots$. Therefore $A U B$ will also be a countably infinite set.

	\item If $A$ is countably infinite and $B$ is uncountably infinite, then $A \cup B$ is countable. \textbf{False.} The union of a countably infinite set $A$ and an uncountably infinite set $B$ remains uncountably infinite. For instance, if $B$ is uncountably infinite (like $\mathbb{R}$), even after adding the elements of the countably infinite set $A$, the union $A \cup B$ retains the uncountability of $B$.

	\item If $A$ is countably infinite and $B$ is uncountably infinite, then $A \cap B$ is countable. \textbf{True.} Since $A$ is countably infinite, it can have at most countably many elements in common with $B$. Thus, $A \cap B$ can either be finite or countably infinite. However, it cannot be uncountably infinite because that would contradict the property of $A$ being countably infinite. Therefore, $A \cap B$ is countable.

\end{itemize}

\vspace{12pt}



\question[3]{
Consider a square grid with length and width $n$. The bottom left corner is considered position $(0,0)$ and the upper right corner is position $(n,n)$ \emph{(*Note that the first item in the tuple is the square along the horizontal axis and the second element is the index along the vertical axis)}. You can see an example grid below.\\
\\
Our goal is to count the number of unique ways a robot starting at cell $(0,0)$ can reach cell $(n,n)$ by only moving up, down, left, right on the grid on each move. We would like you to do two things:

\begin{enumerate}
	\item In your own words, argue why the given set is infinite (as opposed to finite). This given set is infinite due to the nature of the robots movements, which are restricted to left, right, up, and down on the grid. For example, a valid movement could be 5 steps towards the goal $(n,n)$, the there could be redundant repetitions moving right to left, then left to right (in a back and forth direction), then finally the amount of steps required to reach $(n,n)$. 
	\item Show that the number of unique ways the robot can reach position $(n,n)$ is \emph{countably infinite}. \emph{Hint: Try showing that a superset of this one is countably infinite}.
\end{enumerate}
}

\includegraphics[scale=0.4]{lattice.png}

\vspace{12pt}

\question[3]{
Use a proof by diagonalization to show that the following set is uncountable:\\
\\
$F= \mathcal{P}(\mathbb{N}) $
\\
\\
In other words, prove that the power set of the natural numbers (the set of all subsets of the natural numbers) is uncountable.
}

1. \textbf{Assume the opposite}: Doing a proof by contradiction: We can assume that \( P(\mathbb{N}) \) is countable. Then we can list all elements of \( P(\mathbb{N}) \)  as a sequence:
   \[
   S_1, S_2, S_3, S_4, \dots
   \]
   where each \( S_i \) is a subset of the natural numbers \( \mathbb{N} \).\newline

2. \textbf{Represent each subset as a sequence}: We represent each subset as an infinite sequence of 0's and 1's, where 1 indicates the presence of a natural number in the subset, and 0 indicates its absence. Specifically, each subset \( S_i \) can be represented by its characteristic function. The characteristic function for a subset \( S \subseteq \mathbb{N} \) is a function \( f_S: \mathbb{N} \to \{0, 1\} \) defined as:
   \[
   f_S(n) =
   \begin{cases}
   1 & \text{if } n \in S, \\
   0 & \text{if } n \notin S.
   \end{cases}
   \]
   So as a result, each subset \( S_i \) corresponds to a unique infinite binary sequence, with the \( n \)-th element indicating whether \( n \in S_i \).\newline

   For example:

\[
S_1 \text{ is the set of even numbers:}
\]
\[
S_1 = \{ 2, 4, 6, 8, \dots \}
\]

In binary sequence form, this would be:

\[
S_1: (0, 1, 0, 1, 0, 1, 0, 1, \dots)
\]
where the 1st position corresponds to the number 1, the 2nd to 2, the 3rd to 3, and so on. The sequence has a 1 at even-numbered positions and a 0 elsewhere.\newline


3. \textbf{Construct a new subset}: If we make a new subset \( X \subseteq \mathbb{N} \) which is different from \( S_1, S_2, S_3, \dots \). Then \( X \) is such that:
   \[
   n \in X \quad \text{if and only if} \quad n \notin S_n.
   \]
    \( X \) is constructed by diagonalizing the subsets in the Power Set, this is done by looking at the \( n \)-th subset \( S_n \) and including \( n \) in \( X \) if and only if \( n \notin S_n \).\newline

4. \textbf{Contradiction}: \( X \) is a subset of \( \mathbb{N} \), so it must be one of the subsets in the list, which means \( X = S_k \) for some \( k \in \mathbb{N} \). However, by the construction of \( X \), we know that:
   \[
   k \in X \quad \text{if and only if} \quad k \notin S_k.
   \]
   This is a contradiction because it implies that \( k \in X \) and \( k \notin X \), which is impossible.\newline

5. \textbf{Conclusion}: Since assuming that \( P(\mathbb{N}) \) is countable led to a contradiction, we conclude that \( P(\mathbb{N}) \) must be uncountable.
\vspace{12pt}



\end{document}
