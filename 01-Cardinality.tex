\documentclass[12pt]{article}
\usepackage[top=1in,bottom=1in,left=0.75in,right=0.75in,centering]{geometry}
\usepackage{fancyhdr}
\usepackage{epsfig}
\usepackage[pdfborder={0 0 0}]{hyperref}
\usepackage{palatino}
\usepackage{wrapfig}
\usepackage{lastpage}
\usepackage{color}
\usepackage{ifthen}
\usepackage[table]{xcolor}
\usepackage{graphicx,type1cm,eso-pic,color}
\usepackage{hyperref}
\usepackage{amsmath}
\usepackage{wasysym}
\usepackage{amsfonts}

\def\course{CS 3120: Discrete Math and Theory II}
\def\homework{Set Cardinality}
\def\semester{Fall 2024}

\newboolean{solution}
\setboolean{solution}{false}

% add watermark if it's a solution exam
% see http://jeanmartina.blogspot.com/2008/07/latex-goodie-how-to-watermark-things-in.html
\makeatletter
\AddToShipoutPicture{%
\setlength{\@tempdimb}{.5\paperwidth}%
\setlength{\@tempdimc}{.5\paperheight}%
\setlength{\unitlength}{1pt}%
\put(\strip@pt\@tempdimb,\strip@pt\@tempdimc){%
\ifthenelse{\boolean{solution}}{
\makebox(0,0){\rotatebox{45}{\textcolor[gray]{0.95}%
{\fontsize{5cm}{3cm}\selectfont{\textsf{Solution}}}}}%
}{}
}}
\makeatother

\pagestyle{fancy}

\fancyhf{}
\lhead{\course}
\chead{Page \thepage\ of \pageref{LastPage}}
\rhead{\semester}
%\cfoot{\Large (the bubble footer is automatically inserted into this space)}

\setlength{\headheight}{14.5pt}

\newenvironment{itemlist}{
\begin{itemize}
\setlength{\itemsep}{0pt}
\setlength{\parskip}{0pt}}
{\end{itemize}}

\newenvironment{numlist}{
\begin{enumerate}
\setlength{\itemsep}{0pt}
\setlength{\parskip}{0pt}}
{\end{enumerate}}

\newcounter{pagenum}
\setcounter{pagenum}{1}
\newcommand{\pageheader}[1]{
\clearpage\vspace*{-0.4in}\noindent{\large\bf{Page \arabic{pagenum}: {#1}}}
\addtocounter{pagenum}{1}
\cfoot{}
}

\newcounter{quesnum}
\setcounter{quesnum}{1}
\newcommand{\question}[2][??]{
\begin{list}{\labelitemi}{\leftmargin=2em}
\item [\arabic{quesnum}.] {} {#2}
\end{list}
\addtocounter{quesnum}{1}
}


\definecolor{red}{rgb}{1.0,0.0,0.0}
\newcommand{\answer}[2][??]{
\ifthenelse{\boolean{solution}}{
\color{red} #2 \color{black}}
{\vspace*{#1}}
}

\definecolor{blue}{rgb}{0.0,0.0,1.0}

\begin{document}

\section*{\homework}


\question[3]{
For each of the following claims, state whether it is true or false and then prove your assertion.
}

\begin{itemize}
	\item All finite sets have an \emph{injection} to $\mathbb{N}$ \textbf{True}. A finite set $S$ with $n$ elements can be injected into $\mathbb{N}$ by mapping each element $s_i \in S$ to its position $i$ in the sequence $1, 2, \ldots, n$. This function is clearly injective as no two different elements in $S$ map to the same natural number.
	\item All finite sets have a \emph{surjection} to $\mathbb{N}$
    \textbf{False.} A finite set has a finite number of elements, while $\mathbb{N}$ is infinite. A surjective function requires every element in $\mathbb{N}$ to be the image of some element in the finite set, which is impossible because there will always be natural numbers that are not mapped to.
	\item If $A$ is a countably infinite set (i.e., $|A|=|\mathbb{N}|$) and $B$ is a also a countably infinite set (i.e., $|B| = |\mathbb{N}|$), then $A \cup B$ is also countable.\textbf{True.} A and B are countably infinite sets, so the Union of these two sets A and B will produce another countably infinite set. For example, we can count them by mapping values in the natural numbers to alternating values in each set, counting $a1,b1,a2,b2, \ldots$. Therefore $A U B$ will also be a countably infinite set.

	\item If $A$ is countably infinite and $B$ is uncountably infinite, then $A \cup B$ is countable. \textbf{False.} The union of a countably infinite set $A$ and an uncountably infinite set $B$ remains uncountably infinite. For instance, if $B$ is uncountably infinite (like $\mathbb{R}$), even after adding the elements of the countably infinite set $A$, the union $A \cup B$ retains the uncountability of $B$.

	\item If $A$ is countably infinite and $B$ is uncountably infinite, then $A \cap B$ is countable. \textbf{True.} Since $A$ is countably infinite, it can have at most countably many elements in common with $B$. Thus, $A \cap B$ can either be finite or countably infinite. However, it cannot be uncountably infinite because that would contradict the property of $A$ being countably infinite. Therefore, $A \cap B$ is countable.

\end{itemize}

\vspace{12pt}



\question[3]{
Consider a square grid with length and width $n$. The bottom left corner is considered position $(0,0)$ and the upper right corner is position $(n,n)$ \emph{(*Note that the first item in the tuple is the square along the horizontal axis and the second element is the index along the vertical axis)}. You can see an example grid below.\\
\\
Our goal is to count the number of unique ways a robot starting at cell $(0,0)$ can reach cell $(n,n)$ by only moving up, down, left, right on the grid on each move. We would like you to do two things:

\begin{enumerate}
	\item In your own words, argue why the given set (the set of possible paths to the goal) is infinite (as opposed to finite).\\
\\
 The given set (the set of possible paths to the goal) is infinite (as opposed to finite) due to the unrestricted nature of the robot's movements in terms of the number of moves and reversals it can make. While the robot is confined in the directions it is able to move, as it can reach cell $(n,n)$ by only moving up, down, left, or right on the grid on each given move, it is free to move back and forth between cells on the grid indefinitely until it reaches cell $(n,n)$. Accordingly, given this lack of restriction on the number of moves the robot can make on its path to cell $(n,n)$ (and the fact that there is no explicit requirement for the robot to progress directly towards cell $(n,n)$), the robot could take arbitrarily long paths to reach the final position, thereby yielding an unbounded number of total possible paths to the goal. To further illustrate this idea, we can look at the following example: consider the simple scenario where $n$ (the length and width of the square grid) is $1$, such that cell $(n,n)$ is equal to cell $(1,1)$; even in this case, the robot could make an indefinite number of moves in going back and forth between cells $(0, 0)$ and $(1, 0)$ (where it is moving side to side along the horizontal axis repeatedly) and/or cells $(0, 0)$ and $(0, 1)$ (where it is moving up and down along the vertical axis repeatedly) before ultimately reaching cell $(1,1)$. Thus, the given set of all possible paths to the goal in this case is infinite.\\
 
	\item Show that the set of unique ways the robot can reach position $(n,n)$ is \emph{countably infinite}. \emph{Hint: Try showing that a superset of this one is countably infinite}.

\includegraphics[scale=0.4]{lattice.png}\\

To show that the set of unique ways the robot can reach position $(n,n)$ is countably infinite, we can proceed by showing that a superset of this set is countably infinite. In this scenario, we can consider a superset of the given set to be the set of all possible paths that the robot can take from $(0, 0)$ on the grid, including paths that may never reach position $(n,n)$, and, instead, either yield some form of cycle or visit any number of intermediate cells indefinitely. Seeing as each of these paths are fundamentally any sequence of upwards, downwards, leftwards, and rightwards movements, we can more formally express each of these paths as a string over a finite alphabet $\{U, D, L, R\}$, where $U$ represents moving up, $D$ represents moving down, $L$ represents moving left, and $R$ represents moving right.\\
\\
Accordingly, any path that the robot can take from $(0, 0)$ on the grid, including one that may never reach position $(n,n)$ (which, as we outlined earlier, corresponds to any sequence of moves present in the aforementioned alphabet) can be represented as a unique string of arbitrary finite length, where, importantly, each of these strings corresponds to a unique path. More formally, then, we can write this superset of the given set (representing the set of all possible paths that the robot can take from $(0, 0)$ on the grid, including paths that may never reach position $(n,n)$) as $\{U, D, L, R\}^*$.\\
\\
Understanding this, we can construct a mapping to "represent" these path strings with the natural numbers in a similar fashion to the manner in which we showed that the set $\{0, 1\}^*$ is countable in class, which involved constructing a binary tree mapping between the binary strings and natural numbers in a bijective manner, such that different strings map to different numbers (which, particular to the case of the tree structure, means both that different strings map to different nodes in the tree, and no two nodes in the tree have the same index), ensuring that the mapping is injective, and every number appears across the tree nodes, ensuring that the mapping is surjective. Comparably, considering that the superset in this case is represented as $\{U, D, L, R\}^*$, with the alphabet having four characters (as opposed to the two characters pertinent to the previously described binary alphabet), we can construct a quadtree mapping between the path strings and natural numbers in a bijective manner, where each node will have four children to represent each of the four possible movements at any given step (as opposed to the binary tree and its structure of each node having two children to represent each of the two possible binary digits at any given step).\\
\\
While this construction of the quadtree is slightly different from the binary tree construction that was covered in class, the fundamental nature and process are the same; using the quadtree, we can ensure that different strings map to different numbers (which, particular to the case of the tree structure, means both that different strings map to different nodes in the tree, and no two nodes in the tree have the same index), ensuring that the mapping is injective, and every number appears across the tree nodes, ensuring that the mapping is surjective. Thus, because this construction ensures a bijection between the superset of the given set (where the superset is the set of possible paths that the robot can take from $(0, 0)$ on the grid, including one that may never reach position $(n,n)$) and the natural numbers, we can conclude that the superset of the given set is countably infinite.\\
\\
Furthermore, by showing that the superset of the given set (where the given set is the set of unique ways the robot can reach position $(n,n)$) is countably infinite, we can conclude that the set of unique ways the robot can reach position $(n,n)$ must also be countably infinite, as it is the subset of a countably infinite superset.
 
\end{enumerate}
}



\vspace{12pt}

\question[3]{
Use a proof by diagonalization to show that the following set is uncountable:\\
\\
$F= \mathcal{P}(\mathbb{N}) $
\\
\\
In other words, prove that the power set of the natural numbers (the set of all subsets of the natural numbers) is uncountable.
}

1. \textbf{Assume the opposite}: Doing a proof by contradiction: We can assume that \( P(\mathbb{N}) \) is countable. Then we can list all elements of \( P(\mathbb{N}) \)  as a sequence:
   \[
   S_1, S_2, S_3, S_4, \dots
   \]
   where each \( S_i \) is a subset of the natural numbers \( \mathbb{N} \).\newline

2. \textbf{Represent each subset as a sequence}: We represent each subset as an infinite sequence of 0's and 1's, where 1 indicates the presence of a natural number in the subset, and 0 indicates its absence. Specifically, each subset \( S_i \) can be represented by its characteristic function. The characteristic function for a subset \( S \subseteq \mathbb{N} \) is a function \( f_S: \mathbb{N} \to \{0, 1\} \) defined as:
   \[
   f_S(n) =
   \begin{cases}
   1 & \text{if } n \in S, \\
   0 & \text{if } n \notin S.
   \end{cases}
   \]
   So as a result, each subset \( S_i \) corresponds to a unique infinite binary sequence, with the \( n \)-th element indicating whether \( n \in S_i \).\newline

   For example:

\[
S_1 \text{ is the set of even numbers:}
\]
\[
S_1 = \{ 2, 4, 6, 8, \dots \}
\]

In binary sequence form, this would be:

\[
S_1: (0, 1, 0, 1, 0, 1, 0, 1, \dots)
\]
where the 1st position corresponds to the number 1, the 2nd to 2, the 3rd to 3, and so on. The sequence has a 1 at even-numbered positions and a 0 elsewhere.\newline


3. \textbf{Construct a new subset}: If we make a new subset \( X \subseteq \mathbb{N} \) which is different from \( S_1, S_2, S_3, \dots \). Then \( X \) is such that:
   \[
   n \in X \quad \text{if and only if} \quad n \notin S_n.
   \]
    \( X \) is constructed by diagonalizing the subsets in the Power Set, this is done by looking at the \( n \)-th subset \( S_n \) and including \( n \) in \( X \) if and only if \( n \notin S_n \).\newline

4. \textbf{Contradiction}: \( X \) is a subset of \( \mathbb{N} \), so it must be one of the subsets in the list, which means \( X = S_k \) for some \( k \in \mathbb{N} \). However, by the construction of \( X \), we know that:
   \[
   k \in X \quad \text{if and only if} \quad k \notin S_k.
   \]
   This is a contradiction because it implies that \( k \in X \) and \( k \notin X \), which is impossible.\newline

5. \textbf{Conclusion}: Since assuming that \( P(\mathbb{N}) \) is countable led to a contradiction, we conclude that \( P(\mathbb{N}) \) must be uncountable.
\vspace{12pt}



\end{document}
